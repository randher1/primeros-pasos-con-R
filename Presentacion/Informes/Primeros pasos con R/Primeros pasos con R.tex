\documentclass[12pt,a4paper]{article}
\usepackage[left=2.54cm, right=2.54cm, top=2.54cm, bottom=2.54cm]{geometry}
\usepackage[utf8]{inputenc}
\usepackage[T1]{fontenc}
\usepackage[spanish,es-tabla]{babel}
\usepackage{booktabs}
\usepackage{svg}
\usepackage{amsmath}
\usepackage{amsfonts}
\usepackage{amssymb}
\usepackage{graphicx}
\usepackage{multicol}
\usepackage{changepage}
\usepackage{float}
\usepackage{url}
\usepackage{natbib}
\usepackage{multicol}
\usepackage{color}
\usepackage[sc]{mathpazo}
\usepackage{multicol}
\usepackage{titling}
\usepackage{titlesec}
\usepackage{listings}
\usepackage{ragged2e} % Añade el paquete ragged2e
\graphicspath{./imagenes/}
\DeclareGraphicsExtensions{.eps}
\usepackage[colorlinks = true,
linkcolor = black,
urlcolor = blue,
citecolor = black ]{hyperref}
\renewcommand{\bibname}{Bibliografía}
\renewcommand{\baselinestretch}{1.5}

\setlength{\droptitle}{-4.5\baselineskip}
\title{   \begin{center}\rule{0.9\textwidth}{0.1mm} \end{center}
    {\Huge\textbf{Primeros pasos con R en el análisis de datos.}}
    \begin{center}\rule{0.9\textwidth}{0.1mm} \end{center}
}
\author{\textsc{Randolf Herrera Rincon}
    \thanks{Economista - Universidad de La Guajira}\\
    \normalsize 
    \href{mailto:randolfherrerarincon@gmail.com}{randolfherrerarincon@gmail.com}   
}


\lstdefinestyle{Rstyle}{
	language=R,
	basicstyle=\small\ttfamily,
	keywordstyle=\color{blue},
	stringstyle=\color{red},
	commentstyle=\color{green},
	showstringspaces=false,
	breaklines=true,
	frame=none, % Cambiado de frame=tb a frame=none
	backgroundcolor=\color{gray!10},
	captionpos=b,
	tabsize=2,
	rulecolor=\color{black},
	morekeywords={data.frame},
}

% Cambiar "Listing" por "Fragmento de código" en los captions
\renewcommand\lstlistingname{Fragmento de código}


\begin{document}
\maketitle
\tableofcontents
\newpage
\section{Introducción}
\newpage
\section{Carga de archivos}
\subsection{Archivos .csv}
Desde el paquete base de r encontramos la función $read.csv()$ \citep{baseR}
\begin{lstlisting}[style=Rstyle, caption={Lectura de datos en R}, label=lst:readcsv]
	Data <- read.csv("data.csv")
\end{lstlisting}



\begin{table}[!htbp] \centering 
	\caption{Comparación de modelos} 
	\label{} 
	\begin{tabular}{@{\extracolsep{5pt}}lccc} 
		\\[-1.8ex]\hline 
		\hline \\[-1.8ex] 
		& \multicolumn{3}{c}{\textit{Dependent variable:}} \\ 
		\cline{2-4} 
		\\[-1.8ex] & \multicolumn{3}{c}{wage} \\ 
		\\[-1.8ex] & (1) & (2) & (3)\\ 
		\hline \\[-1.8ex] 
		Constant & 6.23$^{***}$ & 4.50$^{***}$ & 5.89$^{**}$ \\ 
		& (0.80) & (1.03) & (2.83) \\ 
		& & & \\ 
		educ & 0.54$^{***}$ & 0.55$^{***}$ & 0.23 \\ 
		& (0.07) & (0.07) & (0.41) \\ 
		& & & \\ 
		I(educ$\hat{\mkern6mu}$2) &  &  & 0.01 \\ 
		&  &  & (0.02) \\ 
		& & & \\ 
		exper &  & 0.20$^{**}$ & 0.34 \\ 
		&  & (0.08) & (0.44) \\ 
		& & & \\ 
		I(exper$\hat{\mkern6mu}$2) &  &  & $-$0.01 \\ 
		&  &  & (0.03) \\ 
		& & & \\ 
		\hline \\[-1.8ex] 
		Observations & 100 & 100 & 100 \\ 
		\hline 
		\hline \\[-1.8ex] 
		Nota: & \multicolumn{3}{r}{$^{*}$p$<$0.1; $^{**}$p$<$0.05; $^{***}$p$<$0.01} \\ 
	\end{tabular} 
\end{table}  


\begin{table}[!htbp] \centering 
	\caption{Comparación de modelos} 
	\label{} 
	\begin{tabular}{@{\extracolsep{5pt}}lccc} 
		\\[-1.8ex]\hline 
		\hline \\[-1.8ex] 
		& \multicolumn{3}{c}{\textit{Dependent variable:}} \\ 
		\cline{2-4} 
		\\[-1.8ex] & \multicolumn{3}{c}{wage} \\ 
		\\[-1.8ex] & (1) & (2) & (3)\\ 
		\hline \\[-1.8ex] 
		Constant & 6.23$^{***}$ & 4.50$^{***}$ & 5.89$^{**}$ \\ 
		& (0.80) & (1.03) & (2.83) \\ 
		& & & \\ 
		educ & 0.54$^{***}$ & 0.55$^{***}$ & 0.23 \\ 
		& (0.07) & (0.07) & (0.41) \\ 
		& & & \\ 
		I(educ$\hat{\mkern6mu}$2) &  &  & 0.01 \\ 
		&  &  & (0.02) \\ 
		& & & \\ 
		exper &  & 0.20$^{**}$ & 0.34 \\ 
		&  & (0.08) & (0.44) \\ 
		& & & \\ 
		I(exper$\hat{\mkern6mu}$2) &  &  & $-$0.01 \\ 
		&  &  & (0.03) \\ 
		& & & \\ 
		\hline \\[-1.8ex] 
		Observations & 100 & 100 & 100 \\ 
		\hline 
		\hline \\[-1.8ex] 
		\textit{Note:}  & \multicolumn{3}{r}{$^{*}$p$<$0.1; $^{**}$p$<$0.05; $^{***}$p$<$0.01} \\ 
	\end{tabular} 
\end{table} 

\newpage
\bibliography{biblio}
\bibliographystyle{apalike}     
\end{document}
