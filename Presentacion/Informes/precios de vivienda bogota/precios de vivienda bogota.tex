\documentclass[12pt,a4paper]{article}
\usepackage[left=2.54cm, right=2.54cm, top=2.54cm, bottom=2.54cm]{geometry}
\usepackage[utf8]{inputenc}
\usepackage[T1]{fontenc}
\usepackage[spanish,es-tabla]{babel}
\usepackage{booktabs}
\usepackage{svg}
\usepackage{amsmath}
\usepackage{amsfonts}
\usepackage{amssymb}
\usepackage{graphicx}
\usepackage{multicol}
\usepackage{changepage}
\usepackage{float}
\usepackage{url}
\usepackage{natbib}
\usepackage{multicol}
\usepackage{color}
\usepackage{colortbl}
\usepackage[sc]{mathpazo}
\usepackage{multicol}
\usepackage{titling}
\usepackage{titlesec}
\usepackage{listings}
\usepackage{ragged2e} % Añade el paquete ragged2e
\graphicspath{./imagenes/}
\DeclareGraphicsExtensions{.eps}
\usepackage[colorlinks = true,
linkcolor = black,
urlcolor = blue,
citecolor = black ]{hyperref}
\renewcommand{\bibname}{Bibliografía}
\renewcommand{\baselinestretch}{1.5}

\setlength{\droptitle}{-4.5\baselineskip}
\title{   \begin{center}\rule{0.9\textwidth}{0.1mm} \end{center}
    {\Huge\textbf{Precios para la valoración de vivienda en Bogotá.}}
    \begin{center}\rule{0.9\textwidth}{0.1mm} \end{center}
}
\author{\textsc{Randolf Herrera Rincon}
    \thanks{Economista - Universidad de La Guajira}\\
    \normalsize 
    \href{mailto:randolfherrerarincon@gmail.com}{randolfherrerarincon@gmail.com}   
}


\lstdefinestyle{Rstyle}{
	language=R,
	basicstyle=\small\ttfamily,
	keywordstyle=\color{blue},
	stringstyle=\color{red},
	commentstyle=\color{green},
	showstringspaces=false,
	breaklines=true,
	frame=none, % Cambiado de frame=tb a frame=none
	backgroundcolor=\color{gray!10},
	captionpos=b,
	tabsize=2,
	rulecolor=\color{black},
	morekeywords={data.frame},
}

% Cambiar "Listing" por "Fragmento de código" en los captions
\renewcommand\lstlistingname{Fragmento de código}


\begin{document}
\maketitle
\tableofcontents
\newpage

\section{Obtención de datos}
Los datos se obtuvieron, de la plataforma kaggle, en el siguiente enlace: \href{https://www.kaggle.com/datasets/pablobravo73/real-estate-bogota}{Análisis del mercado inmobiliario en Bogotá} y ``contiene información sobre inmuebles en Bogotá, Colombia, incluyendo el tipo de propiedad, descripción, cantidad de habitaciones y baños, área, barrio, UPZ (Unidad de Planeamiento Zonal) y valor" \citep{Camacho2024}
\section{Estimación}
Se espera evaluar el siguiente modelo:

\begin{equation}
	\log(\widehat{Valor\_numerico})_{i}=\widehat{\beta{0}} + \widehat{\beta_{1}Banos}_{i} + \beta_{2}\log(\widehat{Area})_{i} + \beta_{3}\widehat{Habitaciones}_{i} + \epsilon
\end{equation}

Donde se espera que $\beta_{1}, \beta_{2}, \beta_{3} > 0$

\section{Cumplimiento de los supuestos econométricos y medidas a corregir}

\begin{enumerate}
	\item Normalidad de residuos
	\begin{itemize}
		\item Prueba de Jarque-Bera
	\end{itemize}
\end{enumerate}

\newpage
\bibliography{biblio}
\bibliographystyle{apalike}     
\end{document}
